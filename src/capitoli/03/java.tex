\section{Installare Java}

    In questo capitolo viene spiegato come installare Java, va bene una qualunque distribuzione (OpenJdk, Corretto etc...) e qualsiasi \textbf{versione maggiore o uguale alla 21}. Tenete a mente 
    che sui computer del laboratorio al momento della scrittura della guida è installata la versione 21, perciò i comandi si riferiscono a quella versione.

    \subsection{Linux}
        Per installare Java su linux è sufficiente utilizzare il gestore dei pacchetti della propria distribuzione, nel caso di\\
        \textbf{Debian, Ubuntu e derivate:}
        \begin{ret}
            \texttt{\$ sudo apt install openjdk-21-jdk -y}
        \end{ret}
        
        \textbf{Fedora, RHLE e derivate:}
        \begin{ret}
            \texttt{\$ sudo dnf install java-21-openjdk -y}
        \end{ret}

        \textbf{Arch e derivate:}
        \begin{ret}
            \texttt{\$ sudo pacman -S jdk21-openjdk}
        \end{ret}
        per tutte le altre distribuzione, vi invito a cercare su internet prestando attenzione ad installare il \textbf{JDK} e non il JRE.
    
    \subsection{Windows}
        Su Windows useremo la distribuzione di Java di Amazon, chiamata Corretto, non ha nessuna differenza rispetto a OpenJdk, almeno per gli utilizzi che ne andremo a fare
        ma è più semplice da trovare e installare.

        per prima cosa scaricate l'installer \url{https://corretto.aws/downloads/latest/amazon-corretto-21-x64-windows-jdk.msi}\\
        nel caso in cui il link non sia più funzionante è sufficiente cercare Corretto jdk 21, aprire la pagina di AWS e scaricare il file .msi per Windows x64.
    
    \subsection{MacOS}
    Su MacOS è importante fare attenzione a quale processore è presente nel vostro Mac
    \subsubsection{Chip M}
        Scaricate il file .pkg al seguente indirizzo \url{https://corretto.aws/downloads/latest/amazon-corretto-21-aarch64-macos-jdk.pkg} e installate il JDK.
    \subsubsection{Chip Intel}
        Scaricate il file .pkg al seguente link \url{https://corretto.aws/downloads/latest/amazon-corretto-21-x64-macos-jdk.pkg} e installate il JDK.
    
    \hfill \break
    Nel caso in cui il link per scaricare il pkg non sia più funzionante è sufficiente cercare Corretto jdk 21, aprire la pagina di AWS e scaricare il file .pkg per MacOS prestando 
    attenzione all'architettura del vostro processore, se il vostro Mac ha un chip della \textbf{serie M} allora scaricate la versione \textbf{aarch64} se invece il vostro Mac
    ha un processore \textbf{Intel} scaricate la versione \textbf{x64}.